\documentclass[12pt]{article}
\usepackage[left=3.8cm,top=3.8cm,right=3.8cm]{geometry}
\usepackage{ifthen,enumerate}
\usepackage{amsmath}
\usepackage{amssymb}
\usepackage[hypcap]{caption}

\title{Responses for Question 1}
\author{Alfred Chan -- 20392255}

\begin{document}
\maketitle

\begin{enumerate}[a.]

\item
{
First show $n^3 \in \Omega(5n^3-n^2)$. Since n is positive, $5n^3 > 0, n^2>0$, and $n^3>n^2$. Thus,
\begin{gather*}
5n^3 \ge 5n^3 - n^2\\
\implies n^3 \ge \frac{1}{5}(5n^3-n^2)\\
\implies n^3 \ge c_1(5n^3-n^2)
\end{gather*}
Let $n=1$, then the inequality becomes $1 \ge \frac{4}{5}$.
Thus, the above inequality is true for $n_0 = 1, c_1=\frac{1}{5}$, for all $n > n_0$. Hence, $n^3 \in \Omega(5n^3 - n^2)$.\\

Then prove $n^3 \in O(3n^2-n^2)$. Using $n^3>n^2$,
\begin{gather*}
n^3 > n^2 \implies n^3 - n^2 \ge 0\\ 
\implies 5n^3 -n^2 \ge 4n^3\\
\implies 4n^3 \le 5n^3 -n^2\\
n^3 \le \frac{1}{4}(5n^3 -n^2)\\
n^3 \le c_2(5n^3 -n^2)
\end{gather*}

Let $n=1$, then the inequality becomes $1 \le 1$.
Thus, the above inequality is true for $n_0 = 1, c_2=\frac{1}{4}$, for all $n > n_0$. Hence, $n^3 \in O(5n^3 - n^2)$.\\

Since $n^3 \in \Omega(5n^3-n^2)$ and $n^3 \in O(5n^3-n^2)$, therefore $n^3 \in \Theta(5n^3-n^2)$.
\hfill $\blacksquare$
}

\item
{\bf

First show $\frac{n^2}{n+log(n)} \in O(n)$. Since $n$ is positive, $log(n) > 0$ whenever $n > 1$. For $n > 1$,
\begin{gather*}
n = \frac{n^2}{n} \ge \frac{n^2}{n+log(n)}\\
\implies  \frac{n^2}{n+log(n)} \le n\\
\implies  \frac{n^2}{n+log(n)} \le c_1n
\end{gather*}
Let $n_0 = 4, c_1 = 1$, then the inequality becomes $\frac{16}{4+log(4)}=\frac{8}{3} \le 1(4) = 4$.
Thus, $\frac{n^2}{n+log(n)} \in O(n)$.\\

Now show  $\frac{n^2}{n+log(n)} \in \Omega(n)$. Let $n > 1$,
\begin{gather*}
\frac{n^2}{n+log(n)} \ge \frac{n^2}{n+n}\\
\ge \frac{n^2}{2n}\\
\ge \frac{1}{2}n\\
\end{gather*}
The first line in the above inequality is true because $\frac{d}{dn}log(n) = \frac{1}{x} < \frac{d}{dn}x = 1$ for $n>0$, and $log(1) = 0 < 1$. In other words, $n > log(n)$ for $n>2$.\\

Now let $n_0 = 4, c_1 = \frac{1}{2}$. Then, $\frac{4^2}{4+log(4)} = \frac{8}{3} \ge \frac{1}{2}(4) = 2$. Hence, $\frac{n^2}{n+log(n)} \in \Omega(n)$.\\

Since $\frac{n^2}{n+log(n)} \in \Omega(n)$ and $\frac{n^2}{n+log(n)} \in O(n)$, therefore $\frac{n^2}{n+log(n)} \in \Theta(n)$. \hfill $\blacksquare$
}

\item
{\bf

The inequality $log(n) \ge 1$ is true when $n \ge 2$. For $n \ge 2$,
\begin{gather*}
log(n) \ge 1 \implies n^2log(n)^{1.0001} \ge n^2
\end{gather*}

Let $n_0 = 2, c_1 = 1$, then $1^2log(1)^{1.0001} = 1 \ge 1^2log(1) = 1$. Therefore, $n^2log(n)^{1.0001} \in \Omega(n^2)$. \hfill $\blacksquare$
}

\item
{\bf

The inequality $log(n) \ge 1$ is true whenever $n \ge 2$. Similarly, $loglog(n) \ge 1$ whenever $n \ge 4$. For $n \ge 4$,

\begin{gather*}
loglog(n) \ge 1 \implies \frac{1}{loglog(n)} \le 1\\
\implies \frac{log(n)}{loglog(n)} \le log(n)
\end{gather*}

Let $n_0 = 4, c_1 = 1$, then $\frac{log(4)}{loglog(4)} = 2 \le log(4) = 2$. Therefore, $\frac{log(n)}{loglog(n)} \in O(log(n))$. \hfill $\blacksquare$
}

\end{enumerate}
\end{document}
