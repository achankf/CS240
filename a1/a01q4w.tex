\documentclass[12pt]{article}
\usepackage[left=3.8cm,top=3.8cm,right=3.8cm]{geometry}
\usepackage{ifthen,enumerate}
\usepackage{amsmath}
\usepackage{amssymb}
\usepackage[hypcap]{caption}

\title{Responses for Question 4}
\author{Alfred Chan -- 20392255}

\begin{document}
\maketitle
{\bf
Let $h(x) = \max \{f(n),g(n)\}$ (note: it is MAX).
Since $f(n)$ and $g(n) > 1$ are positive, $h(n)$ is also positive and $h(n) > 1$.
Then let $q(n)$ be the followng:
\begin{gather*}
q(n) = \frac{f(n)g(n)}{f(n)+2g(n)+\log(g(n))}\\
\end{gather*}

As previously explained, since $n > log(n)$ for $n>0$, then $h(n) > log(h(n))$.
By the manipulation of the denomintor, consider the following:

\begin{gather*}
\frac{f(n)g(n)}{h(n)} \ge q(n) \ge \frac{f(n)g(n)}{h(h)+2h(n)+h(n)}\\
\end{gather*}
The above inequality is true because the demoninator, $h(n)$ must be greater than 1, and it has at least one instance of $h(n)$ in the denominator.
Hence,
\begin{gather*}
\frac{f(n)g(n)}{h(n)} \ge q(n) \ge \frac{f(n)g(n)}{4h(n)}\\
\end{gather*}
Recall that $h(n) = \max \{f(n),g(n)\}$.
\begin{gather*}
\implies \min \{f(n),g(n)\} \ge q(n) \ge \frac{1}{4} \min \{f(n),g(n)\}
\end{gather*}
Thus, $q(n) \in \Theta(\min \{f(n),g(n)\})$.
\hfill $\blacksquare$

}

\end{document}

