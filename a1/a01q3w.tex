\documentclass[12pt]{article}
\usepackage[left=3.8cm,top=3.8cm,right=3.8cm]{geometry}
\usepackage{ifthen,enumerate}
\usepackage{amsmath}
\usepackage{amssymb}
\usepackage[hypcap]{caption}

\title{Responses for Question 3}
\author{Alfred Chan -- 20392255}

\begin{document}
\maketitle

\newtheorem{property}{Property}
\begin{enumerate}[a.]
\item
{\bf
Consider $f(n)$, for $n>2$,
\begin{gather*}
f(n) = 7n\log n+\log n > n
\end{gather*}

\begin{property}\label{prop:log<exp}
According to a rule from class, for $k>0,b>0$,
\begin{gather*}
(\log n)^k = o(n^b)
\end{gather*}
\end{property}

So,
\begin{gather*}
(\log n)^{20} \in o(n)
\end{gather*}
But $f(n) > n$, so $g(n)$ is also bounded by $f(n)$. Hence, we get the following:
\begin{gather*}
f(n) \in \omega(g(n))\\
f(n) \in \Omega(g(n))\\
f(n) \notin o(g(n))\\
f(n) \notin O(g(n))\\
f(n) \notin \Theta(g(n))
\end{gather*}
\hfill $\blacksquare$
}

\item
{\bf
First, by property \ref{prop:log<exp},
\begin{gather*}
(log(n))^3 \in o(n^{0.01})
\end{gather*}

So $\exists n_0, \forall c>0, n_0 > n$,
\begin{gather*}
(\log(n))^3 < c n^{0.01}\\
\implies n(\log(n))^3 < c n^{1.01}
\end{gather*}

Therefore,
\begin{gather*}
f(n) \in o(g(n))\\
f(n) \in O(g(n))\\
f(n) \notin \omega(g(n))\\
f(n) \notin \Omega(g(n))\\
f(n) \notin \Theta(g(n))
\end{gather*}

\hfill $\blacksquare$
}

\item
{\bf
Let $k = \sqrt{\log\log n}$.
By property \ref{prop:log<exp}. The next statement is true.
\begin{gather*}
k \in \omega(\log k)\\
\implies k > c\log(k)\\
\implies \sqrt{\log\log n} > c\log(\sqrt{\log\log n})\\
\sqrt{\log\log n} > \frac{c}{2}\log\log\log n\\
\implies \sqrt{\log\log n} > c\log\log\log n\\
\end{gather*}

Hence $f(n) \in \omega(g(n))$. In addition, by the properties that was discussed in class,
\begin{gather*}
f(n) \in \Omega(g(n))\\
f(n) \notin o(g(n))\\
f(n) \notin O(g(n))\\
f(n) \notin \Theta(g(n))
\end{gather*}
\hfill $\blacksquare$
}

\item
{\bf
By high school math, for any $ n \in \mathbb{R}$, $|\sin n| \le 1$; so, $0 \le \sin^2 n \le 1$.
\begin{gather*}
\implies 0 \le f(n) = n(\sin n)^2 \le n
\end{gather*}
Pick $n_0 = \pi, c=1$. Then the above inequality becomes 
\begin{gather*}
f(\pi) = 0 < g(\pi) = \pi
\end{gather*}
Hence, $f(n) \in O(g(n))$.\\
On the other hand, $f(n)$ oscillates between 0 to n, so there is no positive constant that forms the lower bound of $f(n)$. Hence, 
\begin{gather*}
f(n) \notin \Omega(g(n))\\
f(n) \notin \omega(g(n))
\end{gather*}
and $f(n) \notin \Theta(g(n))$, as a consequence of $f(n) \notin \Omega(g(n))$.\\

Simularly, $g(n)$ does not always dominate $f(n)$, since $c\times g(n)$ intersects $f(n)$ indefinitely whenever $c=\frac{1}{2}$. 
In other words, let $n= k\pi + \frac{\pi}{2}, k \in \mathbb{Z}$, then 
\begin{gather*}
f(n) =  k\pi + \frac{\pi}{2} > \frac{1}{2} \times g(n) =  \frac{1}{2} (k\pi + \frac{\pi}{2})
\end{gather*}
 Hence, $f(n) \notin o(g(n))$.
\hfill $\blacksquare$
}

\item
{\bf

\begin{property}
According to a rule from class, for $k>1,a>1$,
\begin{gather*}
n^k = o (a^n)
\end{gather*}
\end{property}

By the above property, $n^3 \in o(\frac{3}{2})^n)$.
So, $\exists n_0>n, \forall c > 0$,
\begin{gather*}
n^3 < c (\frac{3}{2})^n\\
\implies \frac{n^4}{n} < c (\frac{3}{2})^n\\
\implies n^4 2^n < c n 3^n\\
\implies g(n) < cf(n)\\
\implies f(n) > \frac{1}{c}g(n)
\end{gather*}
Hence $f(n) \in \omega(g(n))$. Consequently,
\begin{gather*}
f(n) \in \Omega(g(n))\\
f(n) \notin O(g(n))\\
f(n) \notin o(g(n))\\
f(n) \notin \Theta(g(n))
\end{gather*}
\hfill $\blacksquare$
}

\end{enumerate}
\end{document}
