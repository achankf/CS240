\documentclass[11pt]{article}
\usepackage{fullpage,ifthen,enumerate,algo,url,amsmath}

\begin{document}
%
% Headings
%
\begin{center}
UNIVERSITY OF WATERLOO\\
Cheriton School of Computer Science\\[\baselineskip]
{\bf CS240\hfill Data Structures and Data Management \hfill
Winter 2013}\\[\baselineskip]
{\sc \large ASSIGNMENT 4}\\
(Due: Wednesday, March 13, 2013, 9:30am)\\[2\baselineskip]
\end{center}
%
% Main body
%

%\maketitle
\noindent
Please read \url{http://www.student.cs.uwaterloo.ca/~cs240/w13/guidelines.pdf} for guidelines on submission. 
Problems 1b, 1d, 2a-b and 2d are written problems; submit your solutions as PDF files named {\tt a04w.pdf} for all or individual question files named {\tt a04q1bw.pdf}, ... , {\tt a04q2dw.pdf}. 
Implement your programming solutions in C++ and make sure the code you submit runs in the student environment.
Submit code files {\tt a04q1a.cpp}, {\tt a04q1c.cpp}, and {\tt a04q2c.\{h,cpp\}}.
\noindent

\begin{enumerate}
\item Consider an employee database. 
Each employee is given a sequential unique ID when first hired, which is never re-used. 
Over the years the company has hired 10,000 workers. 
That is to say employee IDs from 0 to 9,999 have been issued. 
However due to retirements and ``redundancies'' there are presently only 40 employees working for the company, with a uniformly random employee ID between 0 and 9,999. 

Jonah decides to hash the key space  $\{0,1,2,\ldots,9999\}$ to the set $\{0,1,2,\ldots,39\}$. 
He proposes as a hash function 

\[ h(EID)=\sum_{i=1 \ldots 4} d_i\]

where $EID=d_1d_2d_3d_4$ are the digits of the employee ID. 

\begin{enumerate}
\item {[20 marks]}   
Write a program that generates 40 unique random employee IDs between 0 and 9999. 
Create a table of 40 entries and hash the employee IDs into it using separate chaining. 
Then print for each entry the number of keys contained in that entry. 
E.g.,

\begin{verbatim}
0 - 0
1 - 0
2 - 0
3 - 4
4 - 1
.
.
39 - 0
\end{verbatim}

For testing purposes your program should look for a command line switch "-t" stored in argv{[1]}.
If this switch is present, the program should read 40 IDs from the standard input instead (e.g., using scanf).
Submit the code for this program in {\tt a04q1a.cpp}.

\item {[5 marks]}  
Repeat a few times. 
Observe the pattern in the distribution of collisions. 
Explain the pattern obtained.

\item {[5 marks]} 
Jarmila replaces the hash function above with 

\[ h(EID) = EID \bmod 40\]

Implement this hash function and resubmit your program.
Submit the code for this program in {\tt a04q1c.cpp}.

\item {[4 marks]} 
Run the new program a few times. 
Observe the pattern in the distribution of collisions. 
Explain.
\end{enumerate}

\item 
Suppose that we have a binary search tree with root $t$, right and left subtrees $r$ and $l$.
Let $w_r$ and $w_l$ be the number of nodes in $r$ and $l$, respectively.
Then a binary search tree is {\bf weight-balanced} iff: 
\begin{itemize}
\item it is empty; or, 
\item it is a leaf; or, 
\item $t$ is the parent of exactly one leaf; or, 
\item $r$ and $l$ are weight-balanced and $w_r$ and $w_l$ are within a multiplicative factor of two of each other.
\end{itemize}

Using plain binary search tree insertion on a weight-balanced tree may trigger an imbalance in the weights (i.e., $w_r \leq 2w_l$ and $w_l \leq 2w_r$ before the insertion, but not after).
This is analogous to how plain insertion into an AVL tree may upset the height balance, except that we apply left, right, double left and double right rotations to restore the height. 

\begin{enumerate}
\item {[10 marks]}
Describe an insertion algorithm for weight-balanced trees that uses rotations (similar in nature to AVL rotations) to preserve/restore the weight balance.

\item {[15 marks]} 
Prove that the rotations used above place the tree back in balance after the insertion proper.

\item {[15 marks]}  
The skeleton code in {\tt a04q2c.\{h,cpp\}} implements a simple binary search tree (not necessarily balanced) in C++.
Resubmit both of these files after modifying {\tt BSTNode::Insert} to rebalance the subtrees using the rotation-based algorithm proposed above.
You may add private member functions to the {\tt BSTNode} class, but do not add fields or modify any functions implemented in the header file.

{\tt BSTNode::Insert} must also update the {\tt leftDescendants}, {\tt rightDescendants} and {\tt balance} fields for marking purposes: the {\tt leftDescendants} and {\tt rightDescendants} fields must contain the number of descendants in each of the subtrees, and the {\tt balance} field must contain the ratio between the left and right subtree sizes.
I.e.,  
$$
balance=
\begin{cases}
rightDescendants/leftDescendants & \mbox{if } leftDescendants \neq 0 \\
0 & \mbox{otherwise}
\end{cases}
$$

\item {[BONUS - 10 marks]}
Suppose a student wishes to submit a simple AVL implementation as their solution to {\tt a04q2a-c}.
Prove whether or not such a solution is correct.
\end{enumerate}
\end{enumerate}
\end{document}

