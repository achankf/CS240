\documentclass[12pt]{article}
\usepackage{amsmath}
\usepackage{amssymb}
\usepackage{graphicx}
\usepackage{ifthen,enumerate}
\usepackage{listings}
\usepackage{courier}
\usepackage{pdfpages}
\usepackage{../thm}
\usepackage[left=3.8cm,top=3.8cm,right=3.8cm]{geometry}
\title{Responses for CS240 Assignment 4 Question 1d}
\author{Alfred Chan -- 20392255}

\begin{document}
\maketitle
After several tries, it seems the values are spread pretty equally over the whole range.

{\bf Mathematical Explaination:} Consider the distribution of the hash function.
Notice that the number of possibility of getting $h(EID) = k$,
where $0 \le k < 40$, is the number of values that has the same remainder when they are divided by 40.

Notice that $EID \in \{0,\cdots,9999\}$ which has 10000 {\bf consecutive} values,
and that 40 (the modulus) divides 10000. Hence there are $\frac{10000}{40}=250$ equally-sized partitions,
such that each has the same remainder.

Since the input values are uniformly distributed (assumption), so the probability of having $k$ is $P(h(EID) = k) = \frac{1}{250}$, for $0 \le k < 40$.
Hence, the distribution of $k$ is also uniformly distributed.
\hfill $\blacksquare$
\end{document}
