\documentclass[12pt]{article}
\usepackage{amsmath}
\usepackage{amssymb}
\usepackage{graphicx}
\usepackage{ifthen,enumerate}
\usepackage{listings}
\usepackage{courier}
\usepackage{pdfpages}
\usepackage{qtree}
%\usepackage[usenames,dvipsnames]{xcolor}
\usepackage{../thm}
\usepackage[left=3.8cm,top=3.8cm,right=3.8cm]{geometry}
\title{Responses for CS240 Assignment 4 Question 2a}
\author{Alfred Chan -- 20392255}

\begin{document}
\lstset{
	language=c,
  keywordstyle=\color{blue}\bfseries,
  commentstyle=\color{red}\bfseries,
  stringstyle=\ttfamily\color{red!50!brown},
	numbers=left,
	showspaces=false,
	showstringspaces=false,
	stepnumber=1,
	basicstyle=\ttfamily,
	morekeywords={or ,and, then},
	frame=single,
	breaklines=true,
	tabsize=2
}
\maketitle
This {\bf bottom-up} algorithm works from balancing the the subtree in the insertion path from the newly inserted node, and then up to the root of the subtree, using AVL-style single/double rotation.  
The tricky part, however, is that AVL tree single/double rotations are not enough and there are more cases to think of.\\

\noindent Consider the main idea of the algorithm:\\
\begin{itemize}
\item
{\bf Make sub-subtree balanced and then do double rotation on the subtree.}
Consider a case that adds complexity to the algorithm, with the following insertion sequence: $4, 3, 5, 1, 2$.\\

\Tree[.4 [.3 [.1 nil 2 ] nil ] 5 ]\\ 

There is no way to balance them in 1 AVL rotation.
The idea is to make the heavy subtree (left) balance and then do one AVL double rotation on the target (4).
\\{\centering
$\xrightarrow{\text{Double Rotate Right on 3}}$
\Tree[.4 [.2 1 3 ] 5 ]
$\xrightarrow{\text{Double Rotate Right on 4}}$
\Tree[.3 [.2 1 nil ] [.4 nil 5 ] ] }

The problem with this approach is that there are more cases to consider, though the next idea solves this issue.
\item
{\bf Use mutual recursion to reduce complicated rotation cases to just single/double rotations.}\\
Consider two subroutines: $decideRotateDirection$, which decides which direction shall a node rotate; and $prepareRotateLeft$ (and a symmetric one for Right), which decides which of single/double rotation will the node run.\\

As you saw in the example above, the first double rotation is only a stepping stone for the second double rotation. Notice that after the first rotation the subtree is AVL-balanced but not weight-balanced.\\

Basically, $prepareRotateLeft$ runs either single/double rotation, after which the subtree may or may not be balanced, so it calls $decideRotateDirection$ {\bf again} to decide whether it should rotate.
Notice that 2b) indirectly proves that the mutual recursion terminates.
\end{itemize}

Pseudocode:
\begin{lstlisting}
decideRotateDirection()
	if this node is balanced then return
	if leftDescendants > rightDescendants then // left-heavy
		prepareRotateRight
	else // right-heavy
		prepareRotateLeft
\end{lstlisting}

\begin{lstlisting}
prepareRotateLeft()
	if // this condition takes care of the case with 3 nodes
		(leftDescendants = right->leftDescendants
			and leftDescendants is not 0)
		// this condition prevents the rotations to transfer more nodes that break the balance.  Notice that when it does a double rotate, it rotates right first, which the subsubtree would transfer 1/3 to 2/3 of its nodes (right side) to the right of the subtree (key to the proof for 2b).  Hence the upcoming left rotation would transfer a smaller amount of nodes.
			or right->leftDescendants > right->rightDescendants
		then
			doubleRotateLeft
	else
		singleRotateLeft

	decideRotateDirection // mutual recursive
	// when a node is rotated, some notes may be transfered to the other subtree, which balance the *root* of the subtree, but mess up the left and right subtrees' balance.  Hence left and right need to be rebalanced.
	right->decideRotateDirection
	left->decideRotateDirection
\end{lstlisting}

\begin{lstlisting}
Insert(value)
	// do normal BST insertion...
	decideRotateDirection
	return true
\end{lstlisting}

\hfill $\blacksquare$
\end{document}
