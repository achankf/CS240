\documentclass[12pt]{article}
\usepackage{amsmath}
\usepackage{amssymb}
\usepackage{graphicx}
\usepackage{ifthen,enumerate}
\usepackage{listings}
\usepackage{courier}
\usepackage{pdfpages}
\usepackage{qtree}
\usepackage{../thm}
\usepackage[left=3.8cm,top=3.8cm,right=3.8cm]{geometry}
\title{Responses for CS240 Assignment 4 Question 2b}
\author{Alfred Chan -- 20392255}

\begin{document}
\maketitle
\noindent{\bf Friendly reminder} Please take a cup of coffee before reading. This is the longest proof I have written so far (the last one was from a3q3, 5 pages long). Also, you may want to open 2a) for reference.\\

\noindent{\bf Idea of proof} For any subtree, find the size of the left and right subsubtrees, and then apply lemma \ref{lem:weight_tree_bounded}, which would imply that the tree is balanced.\\

\noindent{\bf Convention} Let lower case letters be a single BSTNode; capital letters be a subtree (may be single node or empty).\\

\input{../lemma/weight_balance_tree_des_constraints.texin}

\begin{thm}
The rebalancing algorithm described in 2a) rebalances the position of the target node correctly.
\end{thm}

\begin{proof}
\noindent{\bf By structural induction} Consider only the left-related rotations, since the cases and arguments for the right are symmetric to those of the left.

\noindent{\bf Base Cases:}
\begin{itemize}
\item
If the subtree has 0 to 2 nodes, then it is balanced by definition.
\item
If the subtree has 3 nodes, then the possible trees are:\\
{\bf Balanced:}\\
\begin{equation}\label{tree:3-bal}
\text{\Tree[.b a c ]}\\ 
\end{equation}
{\bf Unbalanced (right child is right-heavy):}\\
\begin{equation}\label{tree:3-rh}
\Tree[.a nil [.b nil [.c nil nil ] ] ]\\
\end{equation}
\begin{enumerate}
	\item decideRotationDirection: since the tree is right-heavy, run prepareRotateLeft
	\item prepareRotateLeft: leftDescendants is zero and $right\rightarrow rightDescendant > right\rightarrow leftDescendants$ (implies first condition is false); hence, run single left rotation
	\item result in tree \ref{tree:3-bal}
\end{enumerate}

{\bf Unbalanced (right child is left-heavy):}\\
\begin{equation}\label{tree:3-rh}
\Tree[.a nil [.c [.b nil nil ] nil ] ]\\
\end{equation}
This is similar to the above, except that $prepareRotateLeft$ will run double rotation instead. Thus, the resulting tree is tree \ref{tree:3-bal}.\\
\end{itemize}
Therefore, the base case is true.\\

\noindent{\bf Induction Hypothesis:} Suppose a tree $T$ is being built by the insertion and the rebalancing algorithm, such that and it remains weight-balanced.
Let $x$ be a node that has key $val$, $n$ be the size of the subtree that is used in discussion.
Then $T.Insert(val)$ will insert $x$ into $T$, the rebalancing algorithm rebalance all subtrees until it reaches somewhere high in T (root, root's children, etc.):

\begin{enumerate}
\item
\begin{description}
\item[Balanced:]
\begin{equation}\label{tree:gen-bal}
\text{\Tree[.b A C ]}\\ 
\end{equation}
Notice that either A contains $\frac{1}{3}(n-1)$ and B contains $\frac{2}{3}(n-1)$, or A contains $\frac{2}{3}(n-1)$ and B contains $\frac{1}{3}(n-1)$, by definition (multipilcative factor of 2).\\

\item[Unbalanced (right-heavy and child (d) is right-heavy)]
\begin{equation}\label{tree:id-rr}
\text{\Tree[.b A [.d C [.E x ] ] ] }\\
\end{equation}

\begin{enumerate}
\item $decideRotateDirection$ will call $prepareRotateLeft$, since the subtree is right-heavy.
\item Since $x$ breaks the balance and $x$ is a leaf of $E$; Hence, $E$ has more rightDescendants than leftDescendants (first condition false). So $prepareRotateLeft$ will run single rotation.
\item Since the rotations involve transfering of nodes, hence even though $d$ is balanced, $b$ or $E$ may not.
So, $prepareRotateLeft$ recursively calls $decideRotateDirection$ on its children.
Since this proof is proving that $decideRotateDirection$ is correct for the {\bf position} (i.e. the structural location of $b$) of the current target, {\bf as long as it works for the position of the ``current'' node, it will also work on its children}.
\end{enumerate}
\begin{equation}\label{tree:id-rr-bal}
\text{\Tree[.d [.b A C ] [.E x ] ] }\\
\end{equation}

\begin{claim}
Tree \ref{tree:id-rr-bal} is weight-balanced.
\end{claim}
\begin{proof}
Since the addition of $x$ causes rebalancing, the subtree with root $d$ (from tree \ref{tree:id-rr}) has $\frac{2}{3}$ of the nodes ($\frac{2}{3}(n-1)$), and the subtree A has $\frac{1}{3}$ of the nodes ($\frac{1}{3}(n-1)$).

Now determine the size of C and E. Let $k = \frac{2}{3}(n-1)$.
Since $x$ is inserted under $E$ and unbalancing occurs, so $E$ has more children.
\begin{align*}
|C| &= \frac{1}{3}(k-1)\\
	&= \frac{2n-5}{9}\\
|E| &= \frac{2}{3}(k-1)\\
	&= \frac{4n-10}{9}\\
\implies |E| + |x| &= \frac{4n-10}{9} + 1\\
	&= \frac{4n-1}{9}
\end{align*}
After rotation, the size of the right subtree is $w_{r} = |E| + |x|$, and the size of the left subtree is $w_{l} = |b| + |A| + |C|$.
\begin{align*}
w_{l} &= |b| + |A| + |C|\\
	&= 1 + \frac{1}{3}(n-1) + \frac{2n-5}{9}\\
	&= \frac{5n + 1}{9}\\
\implies \frac{1}{2} &\le \frac{w_r}{w_l} \le 2
\end{align*}
Hence, by lemma \ref{lem:weight_tree_bounded}, tree \ref{tree:id-rr-bal} is balanced.
\end{proof}
Notice that the direction of $x$ does not matter, as it is handled by the base case and the inductive hypothesis.
The important thing is that $x$ breaks the balance and the subtree of root $b$  has to be rotated.

\item[Unbalanced (right-heavy and child (d) is left-heavy):]
\begin{equation}
\text{\Tree[.b A [.d [.C x ] E ] ]}
	\xrightarrow{\text{ split up C into two halves }}
\text{\Tree[.b A [.d [.c [.CL1 x ] CR1 ]  E ] ]}
\text{ or }
\text{\Tree[.b A [.d [.c CL2 [.CR2 x ] ]  E ] ]}
\end{equation}
This is the case where $prepareRotateLeft$ calls for double rotation.
\begin{equation}
\xrightarrow{\text{single right rotation on d}}
	\text{\Tree[.b A [.c [.CL1 x ] [.d CR1 E ] ] ]} \text{ or }
	\text{\Tree[.b A [.c CL2 [.d [.CR2 x ] E ] ] ]}
\end{equation}

\begin{equation}\label{tree:gen-rl-bal}
\xrightarrow{\text{single left rotation on b}}
	\text{\Tree[.c [.b A [.CL1 x ] ] [.d CR1 E ] ]} \text{ or }
	\text{\Tree[.c [.b A CL2 ] [.d [.CR2 x ] E ] ]}
\end{equation}

\begin{claim}
Trees in \ref{tree:gen-rl-bal} are balanced.
\end{claim}

\begin{proof}
Since inserting to $C$ causes rotation, hence $C$ has more nodes than $E$; consequently, the subtree from $d$ is more nodes than $A$. 
Let $k = \frac{2}{3}(n-1)$ and $l = \frac{2}{3}(k-1)$. Hence,
\begin{align*}
|A| &= \frac{1}{3}(n-1)\\
	&= \frac{k}{2}\\
|E| &= \frac{1}{3}(k-1)\\
|C| &= l
\end{align*}
For $C$, consider the two cases using the same argument as above:
\begin{align*}
|CR1| &= \frac{1}{3}(l-1)\\
	&= \frac{2k-5}{9}\\
|CL1| &= \frac{2}{3}(l-1)\\
	&= 2|CR1|\\
|CL2| &= |CR1|\\
|CR2| &= |CL1|
\end{align*}
Now consider the ratio for $CL1, CR1$:
\begin{align*}
w_{l1} &= |b| + |A| + |CL1| + |x|\\
	&= |b| + |A| + 2|CR1| + |x|\\
	&= 1 + \frac{k}{2} + \frac{4k-10}{9} + 1\\
	&= \frac{17k+16}{18}\\
w_{r1} &= |d| + |CR1| + |E|\\
	&= 1 + \frac{2k-5}{9} + \frac{1}{3}(k-1)\\
	&= \frac{10k+12}{18}\\
\implies \frac{1}{2} &\le \frac{w_{r1}}{w_{l1}} \le 2
\end{align*}
By lemma \ref{lem:weight_tree_bounded}, the first case is balanced. Now consider the ratio for $CL2, CR2$:
\begin{align*}
w_{l2} &= |d| + |CR2| + |x| + |E|\\
	&= 1 + \frac{4k-10}{9} + 1 + \frac{k-1}{3}\\
	&= \frac{14k-8}{18}\\
w_{r2} &= |b| + |A| + |CL2|\\
	&= 1 + \frac{k}{2} + \frac{2k-5}{9}\\
	&= \frac{13k+8}{18}\\
\implies \frac{1}{2} &\le \frac{w_{r2}}{w_{l2}} \le 2
\end{align*}
By lemma \ref{lem:weight_tree_bounded}, the second case is also balanced. Hence the claim holds.
\end{proof}
\end{description}
\end{enumerate}
Thus all cases are true.
The proof is complete.
\end{proof}
However, I am not done yet, sorry.\\

As I mentioned in the proof, the above theorem only makes sure that the target position is balanced.
However, some nodes that are being transfer during rotation can cause unbalance to the subsubtree.
That is why in the algorithm I have to recurse $decideRotateDirection$ on the left and right subtree.
In paricular, the recursion calls to the children simply clean up the ``mess'' that is created by the rotation. In conclusion, the algorithm works.
\hfill $\blacksquare$
\end{document}
