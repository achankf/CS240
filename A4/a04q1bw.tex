\documentclass[12pt]{article}
\usepackage{amsmath}
\usepackage{amssymb}
\usepackage{graphicx}
\usepackage{ifthen,enumerate}
\usepackage{listings}
\usepackage{courier}
\usepackage{pdfpages}
\usepackage{../thm}
\usepackage[left=3.8cm,top=3.8cm,right=3.8cm]{geometry}
\title{Responses for CS240 Assignment 4 Question 1b}
\author{Alfred Chan -- 20392255}

\begin{document}
\maketitle
After several tries, it seems that the distribution is biased towards the centre.\\

{\bf Mathematical Explaination:} Consider the distribution of the hash function.
Notice that the number of possibility of getting $h(EID) = k$,
where $0 \le k < 40$, is the same as solving the number of turples 
	$(a_1,a_2,a_3,a_4) \in \{1, \cdots, 10\}^4$ such that $a_1 + a_2 + a_3 + a_4 = k$.
Notice that this is a {\bf Partition Problem}.

The generating function for this problem is
\begin{align}
(1+x+x^2+x^3+x^4+x^5+x^6+x^7+x^8+x^9+x^{10})^4
\end{align}
Using Maple to expand and simplify gives the following:
\begin{gather*}
simplify(expand((1+x+x^2+x^3+x^4+x^5+x^6+x^7+x^8+x^9+x^10)^4))\\
= 1+4x+10x^2+ \cdots
	+ 880x^{19} + 891x^{20} + 880x^{21} + \cdots
	+ 4x^{39} + x^{40}
\end{gather*}
The above result means that there are
\begin{itemize}
\item 1 way to sum 4 digits to 0;
\item 4 way to sum 4 digits to 1;
\item 891 way to sum 4 digits to 20, and so on.
\end{itemize}

Notice that the coefficient of the generating function
	grows from 1 to 891 ($x^{0}$ to $x^{20}$),
	and then shrinks from 891 to 1 ($x^{20}$ to $x^{40}$).
This means that {\bf the distribution is biased towards the centre (reaches the maximum when $k=20$)}.
Since the input values to $h$ are randomly generated in uniform distribution (assumption),
it is more likely that $h(EID)$ gets closer to the centre than the ones that are near the end of the distribution.
\hfill $\blacksquare$
\end{document}
