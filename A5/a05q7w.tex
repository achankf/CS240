\documentclass[12pt]{article}
\usepackage{amsmath}
\usepackage{amssymb}
\usepackage{graphicx}
\usepackage{ifthen,enumerate}
\usepackage[usenames,dvipsnames,svgnames,table]{xcolor}
\usepackage{listings}
\usepackage{courier}
\usepackage{pdfpages}
\usepackage{qtree}
\usepackage{../thm}
\usepackage[left=3.8cm,top=3.8cm,right=3.8cm]{geometry}

\newcommand{\node}[2]{\ensuremath{#1\rightarrow(#2)}}

\title{Responses for CS240 Assignment 5 Question 5}
\author{Alfred Chan -- 20392255}

\begin{document}
\maketitle
Since the predicate of the BST focuses on $X$, the y-coordinate may not neccessarily be in order. For example,\\

\Tree[.(5,1) [.(3,100) (2,1000) (4,10) ] [.(7,0) (6,1) (8,2) ] ]\\
is a valid BST described by the question.
Notice that it is balanced on the x-coordinates, but the order of the y-coordinates do not make sense.
Hence, the search will have to run
\begin{enumerate}
\item one pass of search to search for the x-value range.
\item a second pass to filter the desired points based on the y-value.
\end{enumerate}

\noindent Now consider two {\bf sets} of values:
\begin{align}
\exists x_0, x_{n+1}, X &= \{x_i \in \mathbb{R} | 1 \le i \le n \text{ and } x_0 < x_i < x_{n+1} \} \\
\exists y_0, y_1, Y &= \{ y_1 \in \mathbb{R} | y_0 < y_1\}
\end{align}

{\bf In other words, they are two bounded sets with unique elements with size of $|X| = n $ and $|Y| = 1$.}\\

Then the targeted structured, $T$, is a BST of $(x_i,y_0), 1 \le i \le n$,
in which the order is determined by a predicate on $X$, and then on $Y$.
{\bf Hence, $T$ has $n$ points in total.}\\

Then consider to search for $x_0 < x < x_{n+1}$ and $y=y_0$.
Since $min(X) = x_1 > x_0$ and $max(X) = x_n < x_{n+1}$, every point in $T$ will become a candidate for the second pass.
Then, each point that is found in first pass must be compared again to ensure it is within range. Hence, all $n$ points are being searched.
However, no point will match because all $y = y_0$, which is less than $y_1$.
Therefore, the construction of $T$ match the requirement of the question.
\done
\end{document}
