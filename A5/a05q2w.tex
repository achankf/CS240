\documentclass[12pt]{article}
\usepackage{amsmath}
\usepackage{amssymb}
\usepackage{graphicx}
\usepackage{ifthen,enumerate}
\usepackage{listings}
\usepackage{courier}
\usepackage{pdfpages}
\usepackage{qtree}
\usepackage{../thm}
\usepackage[left=3.8cm,top=3.8cm,right=3.8cm]{geometry}
\title{Responses for CS240 Assignment 5 Question 2}
\author{Alfred Chan -- 20392255}

\begin{document}
\lstset{
	language=c,
  keywordstyle=\color{blue}\bfseries,
  commentstyle=\color{red}\bfseries,
  stringstyle=\ttfamily\color{red!50!brown},
	numbers=left,
	showspaces=false,
	showstringspaces=false,
	stepnumber=1,
	basicstyle=\ttfamily,
	morekeywords={or ,and, then, let, end},
	frame=single,
	breaklines=true,
	tabsize=2
}
\maketitle

The function $interpolate(A, l, r, k)$ is
\begin{equation}
interpolate(A, l, r, k) = \ell+\left\lfloor\frac{k-A[\ell]}{A[r]-A[\ell]}(r-\ell)\right\rfloor
\end{equation}

Then the psuedocodes for the algorithm is
\begin{lstlisting}
gallop_search(A, k)
	let cur <- 0
	let prev <- cur

	while cur < size(A) and k > A[cur] then
		prev <- cur
		// extrapolate based on doubling the size of the current index
		cur <- 2*cur+1

		// break otherwise will result in segmentation fault
		if temp > size(A) then break
		cur <- interpolate(A, prev, temp, k)
	end while

	// fix boundary conditions
	if cur > size(A) then
		cur = size(A) - 1

	return binary_search(A, prev, cur, k)
\end{lstlisting}

Notice that during each iteration of galloping, if the condition of the loop is true, then the search can discard everything before $prev$, which is a result of $k > A[cur]$. So instead of extrapolating from 0 to $2 \times cur + 1$, the algorithm extrapolates from $prev$.
\end{document}
