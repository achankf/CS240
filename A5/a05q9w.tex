\documentclass[12pt]{article}
\usepackage{amsmath}
\usepackage{amssymb}
\usepackage{graphicx}
\usepackage{ifthen,enumerate}
\usepackage[usenames,dvipsnames,svgnames,table]{xcolor}
\usepackage{listings}
\usepackage{courier}
\usepackage{pdfpages}
\usepackage{tikz-qtree}
\usepackage{../thm}
\usepackage[left=3.8cm,top=3.8cm,right=3.8cm,paperwidth=20in, paperheight=18in]{geometry}

\title{Responses for CS240 Assignment 5 Question 9}
\author{Alfred Chan -- 20392255}

\begin{document}
\maketitle

{\bf Notation} Due to limitations in \LaTeX's \texttt{tikz-qtree} package and my limited abilities, let's consider the following notation:
\begin{itemize}
\item bold characters (i.e. {\bf A, B}, etc.) are pointers to the start of the suffix that is mapped in the table below. {\bf This node contains data.}
\item normal characters (i.e. a, b, etc.) indicates the {\bf direction} to what the pattern should follow. {\bf The node does not contain any data, only the edge contains the direction.}
\item for the {\bf suffix trees}, each node has a range $[a,b]$, which indicates the suffix that is being matched.
\end{itemize}

Suffix Table:\\
\begin{tabular}{c|r|r}
Linking character & Suffix (for 1a,c) & Suffix (for 1b,d)\\\hline
A & kneel knell knead keel		& 00111000111001\$\\\hline
B & neel knell knead keel			& 0111000111001\$\\\hline
C & eel knell knead keel			& 111000111001\$\\\hline
D & el knell knead keel				& 11000111001\$\\\hline
E & l knell knead keel				& 1000111001\$\\\hline
F & (space) knell knead keel	& 000111001\$\\\hline
G & knell knead keel					& 00111001\$\\\hline
H & nell knead keel						& 0111001\$\\\hline
I & ell knead keel						& 111001\$\\\hline
J & ll knead keel							& 11001\$\\\hline
K & l knead keel							& 1001\$\\\hline
L & (space) knead keel				& 001\$\\\hline
M & knead keel								& 01\$\\\hline
N & nead keel									& 1\$\\\hline
O & ead keel\\\hline
P & ad keel\\\hline
Q & d keel\\\hline
R & (space) keel\\\hline
S & keel\\\hline
T & eel\\\hline
U & el\\\hline
V & l\\\hline
\end{tabular}\\

Position of the characters in \verb+kneel knell knead keel+:\\
\begin{tabular}{|c|c|c|c|c|c|c|c|c|c|c|c|c|c|c|c|c|c|c|c|c|c|}
\hline
k & n & e & e & l & & k & n & e & l & l & & k & n & e & a & d & & k & e & e & l\\\hline
0 & 1 & 2 & 3 & 4 & 5 & 6 & 7 & 8 & 9 & 10 & 11 & 12 & 13 & 14 & 15 & 16 & 17 & 18 & 19 & 20 & 21\\
\end{tabular}


\begin{enumerate}
%\item \input{a05q9aw.texin} \done
\item \input{a05q9bw.texin} \done
%\item \input{a05q9cw.texin} \done
%\item \input{a05q9dw.texin} \done
\end{enumerate}
\end{document}
