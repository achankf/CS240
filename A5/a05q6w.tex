\documentclass[12pt]{article}
\usepackage{amsmath}
\usepackage{amssymb}
\usepackage{graphicx}
\usepackage{ifthen,enumerate}
\usepackage[usenames,dvipsnames,svgnames,table]{xcolor}
\usepackage{listings}
\usepackage{courier}
\usepackage{pdfpages}
\usepackage{qtree}
\usepackage{../thm}
\usepackage[left=3.8cm,top=3.8cm,right=3.8cm]{geometry}

\newcommand{\node}[2]{\ensuremath{#1\rightarrow(#2)}}

\title{Responses for CS240 Assignment 5 Question 6}
\author{Alfred Chan -- 20392255}

\begin{document}
\maketitle

\noindent {\bf Notation} Let $x \rightarrow A$ denotes a node with x-coordinate $x$ and a pointer to a binary search tree $A$ of y-coordinates, in heap-like array representation.
For example, if a node contains \node{5}{2,nil,3,nil,nil,nil,4}, then the node has a x value of 5 and a y-coordinate BST of
\begin{align*}
A = \Tree[.2 nil [.3 nil 4 ] ]
\end{align*}
The node has the following coordinates: $(5,2), (5,3), (5,4)$.\\

\noindent Then the range tree for q6 is the following:

\Tree[.\node{8}{1}
	[.\node{7}{9}
		[.\node{2}{4} \node{1}{12} \node{3}{11} ]
		[.\node{5}{10} \node{4}{3} \node{6}{2} ] ]
	[.\node{12}{14} [.\node{10}{15} \node{9}{5} \node{11}{6} ] [.\node{14}{13} \node{13}{7} \node{15}{8} ] ] ]

\done
\end{document}
