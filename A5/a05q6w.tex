\documentclass[12pt]{article}
\usepackage{amsmath}
\usepackage{amssymb}
\usepackage{graphicx}
\usepackage{ifthen,enumerate}
\usepackage[usenames,dvipsnames,svgnames,table]{xcolor}
\usepackage{listings}
\usepackage{courier}
\usepackage{pdfpages}
\usepackage{qtree}
\usepackage{../thm}
\usepackage[left=3.8cm,top=3.8cm,right=3.8cm,paperwidth=15in, paperheight=11in]{geometry}

\newcommand{\node}[2]{\ensuremath{#1\rightarrow\bigg(#2\bigg)}}

\title{Responses for CS240 Assignment 5 Question 6}
\author{Alfred Chan -- 20392255}

\begin{document}
\maketitle

\noindent {\bf Notation} Let $p \rightarrow A$ denotes a node with point $p = (x,y)$ and a pointer to another binary search tree of points that is sorted by the y-coordinates, in heap-like array representation.
For example, if a node contains \node{(5,2)}{(5,2),nil,(2,3),nil,nil,nil,(12,4)}, then the node has a point $(5,2)$ and points to a BST of
\begin{align*}
A = \Tree[.(5,2) nil [.(2,3) nil (12,4) ] ]
\end{align*}

\noindent Then the range tree for q6 is the following:

\Tree[.\node{(8,1)}{(15,8),(2,4),(1,12),(6,2),(11,6),(5,10),(12,14),(8,1),(4,3),(9,5),(13,7),(7,9),(3,11),(14,13),(10,15)}
	[.\node{(4,3)}{(7,9),(11,3),(3,11),(6,2),(2,4),(5,10),(1,12)}
		[.\node{(2,4)}{(3,11),(2,4),(1,12)} \node{(1,12)}{} \node{(3,11)}{} ]
		[.\node{(6,2)}{(7,9),(6,2),(5,10)} \node{(5,10)}{} \node{(7,9)}{} ] ]
	[.\node{(12,14)}{(15,8),(11,6),(14,12),(9,5),(13,7),(14,13),(10,15)}
		[.\node{(10,15)}{(11,6),(9,5),(10,15)} \node{(9,5)}{} \node{(11,6)}{} ]
		[.\node{(14,13)}{(15,8),(13,7),(14,13)} \node{(13,7)}{} \node{(15,8)}{} ]
	]
]\\
\done
\end{document}
