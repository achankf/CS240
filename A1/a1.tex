\documentclass[12pt]{article}
\usepackage{ifthen,enumerate}

\begin{document}
%
% Headings
%
\begin{center}
UNIVERSITY OF WATERLOO\\
Cheriton School of Computer Science\\[\baselineskip]
{\bf CS240\hfill Data Structures and Data Management \hfill
Winter 2013}\\[\baselineskip]
{\sc \large ASSIGNMENT 1}\\
(Due: Wednesday, January 23, 2013, 9:30am)\\[2\baselineskip]
\end{center}
%
% Main body
%

%\maketitle
\noindent
Please read http://www.student.cs.uwaterloo.ca/~cs240/W13/guidelines.pdf for
guidelines on submission. The problems on this assignment are all are written problems; submit your solutions
electronically as a PDF with file name a01w.pdf using MarkUs. We will also accept individual question files named a01q1w.pdf, a01q2w.pdf, ... , a01q6w.pdf if you wish to submit questions as you complete them.
\noindent
\begin{enumerate}
\item
Prove each of the following statements from first principles (i.e., directly from the definitions, giving $c$ and $n_0$ values explicitly):
(all logarithms in this assignment are assumed to be base 2).
\begin{enumerate}
\item  $n^3$ is $\Theta(5 n^3- n^2)$,
\item  $\frac{n^2}{n+\log n}$ is $\Theta(n)$,
\item  $n^2(\log n)^{1.0001}$ is $\Omega(n^2)$, and
\item  $\log(n)/\log\log(n)$ is $O(\log(n))$.
\end{enumerate}


\item Prove or disprove that if $f_1(n)=O(g(n))$ and $f_2(n)=O(h(n))$ then
$f_1(n)/f_2(n)=O(g(n)/h(n))$.

\item Indicate, for each pair of functions $f(n)$ and $g(n)$
below, whether or not:
\begin{enumerate}[I]
\item  $f(n)$ is $o(g(n))$,
\item  $f(n)$ is $O(g(n))$,
\item  $f(n)$ is $\Omega(g(n))$,
\item  $f(n)$ is $\Theta(g(n))$, and
\item  $f(n)$ is $\omega(g(n))$.
\end{enumerate}
Justify your answers (that is, give the key idea of a proof) using
definitions and properties of asymptotic notations derived in class.
So, for example, if $f(n)$  is $n$ and $g(n)$ is $n \log \log n$, we
would say: $f(n)$ is $o(g(n))$ and $O(g(n))$ but not $\Omega(f(n))$
or $\Theta(f(n)) \quad $ \ldots  and justify each claim.  Now solve the
problem for each of:
\begin{enumerate}
\item $f(n) = 7n \log n + \log n;\quad g(n) = (\log n)^{20}$
\item $f(n) = n^{1.01}; \quad g(n) = n \log^3 n $
\item $f(n) = \sqrt{\log\log n} ;\quad g(n) = \log \log \log n$
\item $f(n) = n (\sin n )^2;\quad g(n) = n$
\item $f(n) = n 3^n ;\quad g(n) = n^4 2^n$
\item $f(n) = n$  if $n$ is prime, and $\log n$ if $n$ is composite; $g(n) = n^2$
\end{enumerate}

\item Prove from definition that $f(n)g(n)/(f(n)+2g(n)+\log (g(n)))$ is $\Theta(\min\{f(n),g(n)\} )$.
Assume that $f$ and $g$ are asymptotically nonnegative.
\item For $n$ a power of two, let $T(n)$ be the number of times the
following method prints x. Give a recurrence for $T$ and solve
to get a closed form.
\begin{verbatim}
proc foo(A[1..n]) # assume n>=1 is a power of two
   print("x")
   if n>1 then
      k := 1
      while k<n do
          k := 4*k
          foo(A[1..n/k])
\end{verbatim}
\item Analyze the running time (in terms of $n$) for each of the following
code fragments using $\Theta$-notation.  Briefly justify the runtime,
and simplify as much as possible.
\begin{enumerate}
\item
\begin{verbatim}
sum := 0
for i from 1 to n*n*n do
   for j from 1 to 2^i do    // 2^i means 2 to the power of i
      sum := sum + 1;
\end{verbatim}

\item
\begin{verbatim}
sum := 0
for i from 1 to log n do   // assume n is a power of 2
   for j from 1 to i do
      sum := sum + 1
\end{verbatim}

\item
\begin{verbatim}
sum := 0
for i from 1 to n^2 do
   for j from n to 2n do
     for k from 1 to 100 do
        sum := sum + 1
\end{verbatim}

\end{enumerate}
\end{enumerate}
\end{document}
