\documentclass[12pt]{article}
\usepackage{amsmath}
\usepackage{amssymb}
\usepackage{graphicx}
\usepackage{ifthen,enumerate}
\usepackage{listings}
\usepackage{courier}
\usepackage{pdfpages}
\usepackage[left=2cm,top=2cm,right=2cm]{geometry}
\title{Responses for CS240 Assignment 2 Question 3}
\author{Alfred Chan -- 20392255}

\begin{document}
\maketitle

\begin{enumerate}
\item 
Since student's priority is determined on a first-come first-serve basis, so time is not needed if the IA uses a first-in first-out queue.
Also, since students' names are unique, so for every sets of students going into the queue, there is a deterministic and fair ordering.
\item After 5 minutes of meeting, a student has to be reinserted in the priority queue with the current time, so the heap has to be adjusted in order to fit the heap properties.
So the heap can either do heapify $O(n)$, or just a normal insert $O(\lg n)$.
On the other hand, re-inserting to the end of a FIFO queue is $O(1)$.
\item Removal of nodes when student in the middle of queue decides to leave:
The IA has to search the entire heap in order to find the location of the student, which is a $O(n)$ operation.
However, deleting a node from a FIFO queue is also $O(n)$.
\item Also, when a student finish asking questions, his/her node has to be removed from the heap.
Deleting from a heap is $O(\lg n)$, whereas deleting from a FIFO queue is just deleting the first element, hence $O(1)$.
\item Both heap and deque can get the next element in $O(1)$.
\end{enumerate}

\begin{tabular}{| l | l | l |}
  \hline
	Operation & Heap & Deque\\
  \hline
	insertOneAtATime & $O(\lg(n))$ & $O(1)$\\
	insertAllAtOnce & with heapify $O(n)$ & $O(n)$\\
	getMin & $O(1)$ & $O(1)$\\
	deleteMin & $O(\lg n)$ & $O(1)$\\
	deleteWithoutKnowingPosition & $O(n)$ & $O(n)$\\
  \hline  
\end{tabular}

\hfill $\blacksquare$
\end{document}
