\documentclass[12pt]{article}
\usepackage{amsmath}
\usepackage{amssymb}
\usepackage{graphicx}
\usepackage{ifthen,enumerate}
\usepackage{listings}
\usepackage{courier}
\usepackage{pdfpages}
\usepackage{../thm}
\usepackage[left=2cm,top=2cm,right=2cm]{geometry}
\title{Responses for CS240 Assignment 2 Question 1}
\author{Alfred Chan -- 20392255}

\begin{document}
\lstset{
	language=bash,
	numbers=left,
	showspaces=false,
	showstringspaces=false,
	stepnumber=1,
	basicstyle=\ttfamily,
	frame=single,
	breaklines=true,
	tabsize=2
}
\maketitle
Let $k = \log n$, and ignore the ``book-keeping'' comparison cases, like if root = NULL, since they are constant. Examine when a heap does comparison:
\begin{itemize}
\item Insertion (one at a time)
	\begin{enumerate}
		\item pushes target to the end of array (no comparison)
		\item bubble up (comparison happens)
	\end{enumerate}
\item Insertion (all at once)
	\begin{enumerate}
		\item pushes all targets to an array (no comparison)
		\item for each element from the bottom, bubble down (comparison happens)
	\end{enumerate}
\item DeleteMax
	\begin{enumerate}
		\item move the root element to a temporary variable (no comparison)
		\item move the last element to the root (no comparison)
		\item bubble down from the root (comparison happens)
	\end{enumerate}
\item GetMax
	\begin{enumerate}
		\item return root element (no comparison)
	\end{enumerate}
\end{itemize}
Therefore, the ``big'' comparisons happen when the heap does ``bubbling'' (up and down).\\

Consider bubbling up.
The target node transverse at most up to the root, while each time the target does a comparison with the current parent node, which decides whether the target should be swaped with its current parent.
Similar explaination applys to bubbling down.
Therefore, bubbling cost $O(\log n) = O(k)$ comparisons, which is ``linear'' with respect to the height of the heap.\\

However, binary search can reduce the number of comparisons to $\log(k) = \log(\log(n)) \in o(\log n)$. 

\begin{lstlisting}
BubbleUp(A, i)
	target = A[i]
	new_position <- binary_heap_search_greater_than(A, i)
	push down every element starting from new_position
	A[new_position] <- target
\end{lstlisting}

Assuming that $binary\_heap\_search\_greater\_than$ searches the branch of a heap ($O(\log n)$) and returns the index of the ancestor nodes that the target node {\bf begins} to be greater than the target node.\\

Therefore, the number of comparisons is $O(\log\log n)$ and number of swaps is $O(\log n)$. Thus, the BubbleUp algorithm is $O(\log n)$. BubbleDown is very simular to BubbleUp, except that new\_position looks for the the nodes that begins to be less than the target, and does push up instead.
\end{document}
