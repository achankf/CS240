\documentclass[12pt]{article}
\usepackage{amsmath}
\usepackage{amssymb}
\usepackage{graphicx}
\usepackage{ifthen,enumerate}
\usepackage{listings}
\usepackage{courier}
\usepackage{pdfpages}
\usepackage{../thm}
\usepackage[left=3.8cm,top=3.8cm,right=3.8cm]{geometry}
\title{Responses for CS240 Assignment 2 Question 4a}
\author{Alfred Chan -- 20392255}

\begin{document}
\lstset{
	language=c,
	numbers=left,
	showspaces=false,
	showstringspaces=false,
	morekeywords={ancestor},
	stepnumber=1,
	basicstyle=\ttfamily,
	frame=single,
	breaklines=true,
	tabsize=2
}
\maketitle

Assume that taking exponents or logs cost constant comparison, and that ``book-keeping'' comparisons are ``out of the question,'' like counters in a for loop.
Let's consider the following formula that allows MaxHeap to find the indice of ancestors with a constant number of comparisons.
\input{../lemma/heap_nth_parent.texin}
The important part of the above proof is that a closed formula exists, which cost $O(1)$ comparison.\\
Now consider a binary-search-based function that looks for the number of ancestors in a given node. Let $start$ be the target node to search from.
\begin{lstlisting}
num_ancestor_to_be_swapped(A[1..n], start)
	low <- 0
	high <- lg(start + 1)

	// look for the count with a binary search on the branch
	while true
		pivot <- (low + high) / 2
		// apply the formula to cur_node
		cur_node <- ancestor(start, pivot)

		if high < low then break

		if A[start] > A[cur_node]
			low <- pivot + 1
		else if A[start] < A[cur_node]
			high <- pivot - 1
		else
			break
	// end of while loop

	// boundary case -- comparing the parent of cur_node with the target node
	if A[ancestor(cur_node,1)] < A[start]
		return pivot + 1
	else
		return pivot
\end{lstlisting}
The number of comparisons of the above function is the following:
\begin{align*}
\text{binary search on a branch of the heap} = O(\lg\lg n)\\
\text{boundary case} = O(1)
\end{align*}
Therefore $num\_ancestor\_to\_be\_swapped$ needs $(\lg\lg n)$ comparisons.\\
Consider $BubbleUp$:
\begin{lstlisting}
BubbleUp(A[1..n], start)
	num <- num_ancestor_to_be_swapped(A, start)
	prev <- start
	for i <- 0 to num
		par <- parent(prev)
		swap elements par and prev in A
		prev <- par
\end{lstlisting}
The number of comparisons of data for $BubbleUp$ is the number in $num\_ancestor\_to\_be\_swapped$, which is $O(\lg\lg n)$ comparisons.

Finally, consider the insertion function:
\begin{lstlisting}
Insert(A[1..n], size, data) // the starter codes provides a MaxHeap class, so in the final implementation, only data is needed in the parameter
	A[size] <- data
	BubbleUp(A, size)
	size++ // assuming that ``size'' is pass-by-reference
\end{lstlisting}
The number of comparisons of data for $Insert$ is the number for $num\_ancestor\_to\_be\_swapped$, which is $O(\lg\lg n)$ comparisons.
But $O(\lg\lg n) \in o(\lg n)$, as required.
\hfill $\blacksquare$\\
{\bf Comment} I finally realize the power of induction.
\end{document}
